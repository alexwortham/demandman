\documentclass[12pt,letterpaper]{article}
\usepackage[ampersand]{easylist}
\usepackage[margin=1in]{geometry}
\usepackage{requirements}
\usepackage{float}
\usepackage{amsmath}
\bibliography{Bibliography.bib}
%% @DocumentRequirement (1.1,"Title")
\title{Requirements}
%% @DocumentRequirement (1.3,"Team #")
\team{1}
%% @DocumentRequirement (1.4,"Team Members")
\author{Alex Chaloux, Alex Wortham, Asanga Bandara, Doug Bouler, Chauncey Meade}
%% @DocumentRequirement (1.5,"Customer")
\customer{EPRI}

\begin{document}

%% @DocumentRequirement (1,"Title Page")
%% @DocumentRequirement (1.2,"Date")
\reqstitlepage

%% @DocumentRequirement (2,"Table of Contents")
\tableofcontents
\clearpage

\section{Project Description}

The power grid in the United States is undergoing significant changes with
the addition of renewable energy sources and the phasing out of fossil
fuels.  A critical component in making these changes is better management of
the demand on the grid.  Currently power companies have no way of controlling
the demand on the grid, and must respond reactively to ensure that the demand is
met.  Responding to the changes in demand comes at significant cost to the power
companies, and these companies are beginning to shift these costs on to their
customers in the form of demand charges.  The technical details of demand
charges can be difficult for some people to understand, and many of those people
will soon be facing higher electric bills which they cannot understand.  The
purpose of our project is to give users a simple way to monitor the electricity
usage of the appliances in their homes and lower their bills by controlling when
major household appliances are run.  Given the time constraints on the project, we will be
simulating AC loads on custom built DC circuits to reduce design complexity.  


\clearpage
%% @DocumentRequirement (4,"Requirements")
\section{Requirements}

%% @DocumentRequirement (4.1,"Numbered list of requirements initially agreed to by team and customer.")
\begin{easylist}[articletoc] \requirements

& Hardware

	&& \req{simulation device} The simulation device will provide outputs
	representative of the following appliances:
		&&& Air conditioning unit
		&&& Clothes dryer
		&&& Dishwasher
		&&& Water heater
	
	&& The monitoring system will detect the electrical DC current for each
	appliance listed in requirement \refreq{simulation device}.
	&& The monitoring system will monitor the current over time through each device
	listed in requirement \refreq{simulation device}.
	&& The control system will record demand usage data to a database.
	
& Software

	&& A web-based user interface will be accessible to the user providing
	the following data:
		&&& The current power usage of each device listed
		in requirement \refreq{simulation device}.
		&&& The current aggregate power usage of all
		devices listed in requirement \refreq{simulation device}.
		&&& A summary of total power usage for the current
		billing cycle.
			&&&& This data will also be represented in a dollar amount.
		&&& Which monitored appliances are currently using power.
			&&&& ``Monitored appliances'' refers to the loads simulated by our simulation
			device and controlled by our control system.
		&&& Which devices can be activated while remaining under the specified demand
		threshold.
			&&&& The controller software will make a best effort attempt to prevent usage
			in excess of this threshold based on current and historical load data.
			&&&& This threshold will be specified by the user via the web-based
			interface.
			
	
	&& The software for this project will be divided into two distinct components.
		&&& The simulation control component.
		&&& The load monitoring and control component.
		
	&& The simulation control component will perform the following functions:
		&&& Maintain a database containing load data for each of the appliance types
		specified by requirement \refreq{simulation device}.
		&&& Start playback of data via command from the user interface.
		&&& Stop playback of data via command from the user interface.
	
	&& The loading monitoring and control component will perform the following
	functions:
		&&& Host a web server for the web-based interface.
		&&& Control relays for each of the appliance types specified by requirement
		\refreq{simulation device}.
		&&& Send start and stop commands from the user interface to the simulation
		control component.
			
\end{easylist}

\section{Summary}

As energy demand continues to rise, the associated costs are becoming an
important problem for many households. The average adult will likely have
difficulty understanding the method through which power companies are passing
this cost on to them, putting efficient management of electrical demand near
impossible. To alleviate this problem, our device seeks to demonstrate a system
through which households can easily manage their demand, thereby reducing their
electrical bill. By giving a single device through which to interface with their
major appliances, our project seeks to make energy conservation a manageable
task for the average household. We are designing a demonstration of the end user
system, as well as a method for simulating real loads for the system to manage.
The above requirements lay out the hardware and software needs of this project.


% % @DocumentRequirement (9,"Signatures of all Team Members and Customer.")
\clearpage
\section{Customer Agreement}

%% @DocumentRequirement (9.1,"Names, signatures, and dates.")

The undersigned agree that the requirements outlined in this document meet all needs for the product.

\signanddate{Harshal Upadhye}
\signanddate{Dr. Mark Dean}
\signanddate{Asanga Bandara}
\signanddate{Doug Bouler}
\signanddate{Alex Chaloux}
\signanddate{Chauncey Meade}
\signanddate{F. Alex Wortham}

%% @DocumentRequirement (9.2,"Dissenting statements (signed) -- if any.")

\end{document}
